\documentclass[12pt]{report}

\usepackage{graphicx} % Required for inserting images
\usepackage[brazil]{babel}
%\usepackage[latin1]{inputenc}
\usepackage{amsmath}
\usepackage{amsfonts}
\usepackage{amssymb}
\usepackage{parskip}
\usepackage{float}
\usepackage{mathtools}
\usepackage[autostyle]{csquotes}
\MakeOuterQuote{"}
\usepackage{geometry}
 \geometry{ a4paper, left=20mm, top=20mm, right=20mm, bottom=20mm}
\usepackage{systeme}
\usepackage{mathtools}
\usepackage{lipsum}
\usepackage{multicol}
\usepackage{fontawesome5} % termômetros
\renewcommand{\baselinestretch}{1.1}

\newcommand{\1}{\faThermometerEmpty}
\newcommand{\2}{\faThermometerQuarter}
\newcommand{\3}{\faThermometerHalf}
\newcommand{\4}{\faThermometerThreeQuarters}
\newcommand{\5}{\faThermometerFull}


\title{Lista}
\author{Eduardo Yukio}
\date{March 2025}

\begin{document}

\begin{center}
{\Large Divisibilidade - 12/03/2025} \\
\vspace{1mm}
Eduardo Yukio G. Ishihara \\
{\footnotesize eduardoyukio.ishihara@usp.br}
\end{center}
\vspace{5mm}

\section*{Números Primos}
Na matemática, definimos os números primos como os números que podem ser divididos apenas por $1$ e por ele mesmo, ou seja todo número primo tem exatamente dois divisores. Convenciona-se que o número $1$ não é um número primo, pois ele tem apenas um divisor: ele mesmo. Os cinco primeiros números primos são $2,3,5,7$ e $11$. Teste e veja se estes números podem ser divididos por qualquer outro número além de $1$ e eles mesmos. Teste, também, se há outros números primos menos do que $11$.

Se um número tem mais do que apenas dois divisores, chamamos ele de \textbf{número composto}. 

No estudo da divisibilidade, os números primos são muito importantes, pois eles "governam" as principais regras de divisibilidade.

\section*{Divisibilidade}
\begin{itemize}
    \item \textbf{Divisibilidade por $1$}: Todo número é divisível por $1$.
    
    \item \textbf{Divisibilidade por $2$}: O número deve ser par, ou seja, o último digito deve ser $0,2,4,6$ ou $8$. \\
    Ex.: $138$ é divisível por $2$, pois termina é par (termina em $8$);\\
    \phantom{Ex.: }$143$ não é divisível por $2$, pois não termina em par ($3$ é ímpar).
    
    \item \textbf{Divisibilidade por $3$}: A soma dos dígitos do número deve ser divisível por 3. \\
    Ex.: $123.456$ é divisível por $3$, pois $1+2+3+4+5+6=21$ e $21$ é divisível por $3$; \\
    \phantom{Ex.: }$4.512.688$ não é divisível por $3$, pois $4+5+1+2+6+8+8=34$ e $34$ não é divisível por $3$. \\
    Note que você pode aplicar o critério da soma dos algarismos mais uma vez sobre o $35$ para verificar se ele é divisível por $3$.
    
    \item \textbf{Divisibilidade por $4$}: Os dois últimos dígitos devem ser divisíveis por $4$.\\
    Ex.: $14.584$ é divisível por $4$, pois $84$ é divisível por $4$ (faça a conta);\\
    \phantom{Ex.: }$412.593$ não é divisível por $4$, pois $93$ não é divisível por $4$.
    
    \item \textbf{Divisibilidade por $5$}: o número deve terminar em $0$ ou $5$.\\
    Ex.:  $4.585$ é divisível por $5$, pois termina em $5$;\\
    \phantom{Ex.: }$84.517$ não é divisível por $5$, pois não termina nem em $5$ nem em $0$.
   
    \item \textbf{Divisibilidade por $6$}: Deve ser divisível por $2$ e por $3$.\\
    Ex.: $123.456$ é divisível por $6$, pois é divisível por $3$ e por $2$ (é par).\\
    \phantom{Ex.: }$4.512.688$ não é divisível por $6$, pois é divisível por $2$, mas não por $3$;\\
    \phantom{Ex.: }$81$ não é divisível por $6$, pois é divisível por $3$, mas não por $2$.\\
    
    
    \item \textbf{Divisibilidade por $8$}:  Os três últimos dígitos devem ser divisíveis por $8$\\
    Ex.: $14.584$ é divisível por $8$, pois $584$ é divisível por $8$ (faça a conta);\\
    \phantom{Ex.: }$15.364$ não é divisível por $8$, pois $364$ não é divisível por $4$.
    
    \item \textbf{Divisibilidade por $9$}: A soma dos digitos do número deve ser divisível por 3. \\
    Ex.: $123.456.789$ é divisível por $9$, pois $1+2+3+4+5+6+7+8+9=45$ e $45$ é divisível por $9$; \\
    \phantom{Ex.: }$4.512.688$ não é divisível por $3$, pois $4+5+1+2+6+8+8=34$ e $34$ não é divisível por $9$;\\
    Note que você pode aplicar o critério da soma dos algarismos mais uma vez sobre o $35$ para verificar se ele é divisível por $9$.
    
    \item \textbf{Divisibilidade por $10$}: O número deve terminar em $0$.\\
    Ex.: $1.548.580$ é divisível por $10$, pois termina em $0$;\\
    \phantom{Ex.: }$154.565$ não é divisível por $10$, pois não termina em $0$.
    
    \item \textbf{Divisibilidade por $11$}: A soma alternada dos dígitos do número deve ser divisível por $11$.\\
    Ex.: $907.071$ é divisível por $11$, pois $9-0+7-0+7-1=22$ e $22$ é divisível por $11$; \\
    \phantom{Ex.: }$291.412$ é divisível por $11$, pois $2-9+1-4+1-2=-11$ e $11$ é divisível por $11$; \\
    \phantom{Ex.: }$249.942$ é divisível por $11$, pois $2-4+9-9+4-2=0$ e $0$ é divisível por $11$; \\
    \phantom{Ex.: }$12.678$ não é divisível por $11$, pois $1-2+6-7+8=6$ e $6$ não é divisível por $11$. \\
    Note que a divisibilidade não é afetada pelo sinal do número ($120$ e $-120$ têm os mesmos divisores, por exemplo) e $0$ é divisível por todos os números.
\end{itemize}

\pagebreak

\begin{center}
{\Large Exercícios}
\end{center}

A lista é extensa, nem todos os itens devem ser feitos na íntegra. Faça os itens até que você tenha certeza de que é capaz de fazer os demais itens sem dificuldades. 

Para ajudá-los, adotei um sistema para classificar a dificuldade dos exercícios que vai de \1 \ (fácil), passa por \2 , \3 , \4 \ e chega a \5 \ (muito difícil). Tal métrica não é absoluta e pode conter erros ou variar de pessoa para pessoa!

\vspace{5mm}

\begin{enumerate}
\item Decomponha os números abaixo em fatores primos e determine se são primos ou compostos. \1
\begin{multicols}{4}
  \begin{enumerate}
    \item 1
    \item 2
    \item 4
    \item 6
    \item 12
    \item 24
    \item 35
    \item 49
    \item 56
    \item 68
    \item 69
    \item 73
    \item 74
    \item 82
    \item 89
    \item 94
    \item 99
    \item 110
    \item 126
    \item 134
    \item 141
    \item 142
    \item 143
    \item 144
    \item 999
    \item \2 27.720
    
    \end{enumerate}
\end{multicols}

\item Determine se os números abaixo são divisíveis por $2,3,4,5,6,8,9,10$ e $11$. De $(a)$ a $(l):$ \1, de $(m)$ a $(z):$ \2.
\begin{multicols}{4}
  \begin{enumerate}
    \item 0  
    \item 1
    \item 10
    \item 100
    \item 1.000
    \item 10.000
    \item 60
    \item 625
    \item 660
    \item 1.250
    \item 2.025
    \item 8.640    
    \item 145.124
    \item 5.702.400
    \item 15.449.013    
    \item 23.022.006
    \item 25.122.025
    \item 111.111.111
    \item 222.222.222
    \item 555.555.555
    \item 777.777.777
    \item 999.999.999
    \item 123.456.789
    \item 228.659.875
    \item 654.896.587
    \item 987.654.321
    \end{enumerate}
\end{multicols}


\item Crie regras de divisibilidade para os números abaixo. Dicas: tente decompô-los em fatores primos e aplique o mesmo princípio da regra de divisibilidade por 6 e o padrão das regras de divisibilidade por $2$, $4$ e $8$. De $(a)$ a $(n):$ \2, $(o):$ \3 e $(p):$ \4.

\begin{multicols}{4}
  \begin{enumerate}
    \item  16
    \item  32
    \item  64
    \item  12
    \item  24
    \item  22
    \item  18
    \item  40
    \item  45
    \item  100
    \item  1.000
    \item  20
    \item  200
    \item  2.000
    \item  36
    \item  $2^n$
    \end{enumerate}

\end{multicols}

\end{enumerate}
\end{document}

