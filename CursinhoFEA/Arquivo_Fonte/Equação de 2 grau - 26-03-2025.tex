\documentclass[12pt]{report}

\usepackage{graphicx} % Required for inserting images
\usepackage[brazil]{babel}
%\usepackage[latin1]{inputenc}
\usepackage{amsmath}
\usepackage{amsfonts}
\usepackage{amssymb}
\usepackage{parskip}
\usepackage{float}
\usepackage{mathtools}
\usepackage[autostyle]{csquotes}
\MakeOuterQuote{"}
\usepackage{geometry}
 \geometry{ a4paper, left=20mm, top=20mm, right=20mm, bottom=20mm}
\usepackage{systeme}
\usepackage{mathtools}
\usepackage{lipsum}
\usepackage{enumitem}
\usepackage{multicol}
\usepackage{multirow}
\usepackage{array}
\newcommand\mc[1]{\multicolumn{1}{c}{#1}}
\usepackage{xfrac}
\usepackage{nicematrix,tikz,booktabs}
\usepackage{fontawesome5} % termômetros
\renewcommand{\baselinestretch}{1.1}

\usepackage{fancyhdr}
\pagestyle{fancy}
\fancyhf{}
\fancyfoot[LO]{\footnotesize {Eduardo Yukio G. Ishihara\\ eduardoyukio.ishihara@usp.br}}
\fancyfoot[C]{\thepage}
\renewcommand{\headrulewidth}{0pt}
\renewcommand{\footrulewidth}{0pt}


\newcommand{\1}{\faThermometerEmpty}
\newcommand{\2}{\faThermometerQuarter}
\newcommand{\3}{\faThermometerHalf}
\newcommand{\4}{\faThermometerThreeQuarters}
\newcommand{\5}{\faThermometerFull}


\title{Lista}

\begin{document}

\begin{center}
{\Large Equação de Segundo Grau - Parte 1} \\ % ALTERAR
\vspace{1mm}
Eduardo Yukio G. Ishihara \\ 19/03/2025 \\ % ALTERAR
{\footnotesize eduardoyukio.ishihara@usp.br \\
Editado em 21/04/2025} % ALTERAR
\end{center}
\vspace{5mm}

\section*{Definição}
Definimos equações de segundo grau, ou equações quadráticas, como todas as equações da forma
$$ a x^2+bx+c=0$$,
Onde:
\begin{itemize}
    \item $x$ é a variável (ou incógnita);
    \item $a,b,c$ são os coeficientes
    \item $a \neq 0$, caso contrário seria uma equação de $1$\textdegree{} grau;
    \item $b,c \in \mathbb{R}$, $b$ e $c$ são números reais quaisquer, até mesmo zero.
\end{itemize}

Para encontrar as soluções da equação, ou suas raízes, temos algumas alternativas já bem conhecidas.

\subsection*{Fórmula de Bhaskara}
O método mais conhecido e usado para encontrar as soluções da equação é a fórmula de \text{Bhaskara}:
$$
\begin{aligned}
    x &= \frac{-b \pm \sqrt{\Delta}}{2a} \\
    \Delta &= b^2-4ac
\end{aligned} $$
Ao aplicar a fórmula de Bhaskara, é interessante notar que há três possíveis casos para o valor de $\Delta$:
\begin{itemize}
    \item $\Delta <0$: não há nenhuma raiz real;
    \item $\Delta = 0$: há duas raízes reais iguais ou, alternativamente, pode-se interpretar de maneira didática que a equação tem apenas uma solução real;
    \item $\Delta >0$: há duas raízes reais distintas.
\end{itemize}
Quando $\Delta<0$, seria necessário calcular a raiz quadrada de um número negativo, o que é impossível quando lidamos apenas com os números reais ($\mathbb{R}$). Quando $\Delta = 0$, somar $\sqrt{0}$ ou subtrair $\sqrt{0}$ não altera o valor de $x$, por isso temos duas raízes reais iguais. Ao calcular as duas raízes, é usual definir $x_1$ e $x_2$ como as duas raízes da equação e tal padrão será mantido neste documento.

O método da fórmula de Bhaskara é muito utilizado no Brasil, pois é um algoritmo fácil de ser decorado, os procedimentos são muito bem definidos e o cálculo do discriminante ($\Delta$) permite ao aluno perceber quantas raízes tem uma equação.

Em contrapartida, para alguns casos, as contas necessárias podem ser muito extensas ou complicadas, quando os coeficientes são números muito grandes ou com muitas casas decimais.

A fórmula de Bhaskara é muito antiga e, ao contrário do que muitas pessoas acreditam, não foi criada pelo matemático indiano Bhaskara II. Uma demonstração da fórmula pode ser encontrada após a seção de exercícios.



\subsubsection*{Exemplo 1.1}

Considere a equação:
$$ x^2 - 5x + 6 = 0 $$

Identificamos os coeficientes:
\begin{itemize}
    \item $a = 1$
    \item $b = -5$
    \item $c = 6$
\end{itemize}

Calculamos o discriminante:
$$ \Delta = b^2 - 4ac = (-5)^2 - 4 \cdot 1 \cdot 6 = 25 - 24 = 1 $$

Aplicando a fórmula de Bhaskara:
$$
x = \frac{-b \pm \sqrt{\Delta}}{2a} = \frac{-(-5) \pm \sqrt{1}}{2 \cdot 1} = \frac{5 \pm 1}{2}
$$

Assim, as duas soluções são:
\[
\begin{aligned}
    x_1 &= \frac{5 - 1}{2} = 2 \\
    x_2 &= \frac{5 + 1}{2} = 3
\end{aligned}
\]

Portanto, as raízes da equação são $x_1 = 2$ e $x_2 = 3$.



\subsubsection*{Exemplo 1.2}
Considere a equação:
$$ x^2 - 6x + 9 = 0 $$

Coeficientes:
\begin{itemize}
    \item $a = 1$
    \item $b = -6$
    \item $c = 9$
\end{itemize}

Calculamos o discriminante:
$$ \Delta = b^2 - 4ac = (-6)^2 - 4 \cdot 1 \cdot 9 = 36 - 36 = 0 $$

Aplicando a fórmula de Bhaskara:
$$
x = \frac{-b \pm \sqrt{0}}{2a} = \frac{6 \pm 0}{2} = 3
$$

Como $\Delta = 0$, temos duas raízes reais iguais:
\[
x_1 = x_2 = 3
\]



\subsubsection*{Exemplo 1.3}

Considere a equação:
$$ x^2 + 2x + 5 = 0 $$

Coeficientes:
\begin{itemize}
    \item $a = 1$
    \item $b = 2$
    \item $c = 5$
\end{itemize}

Calculamos o discriminante:
$$ \Delta = b^2 - 4ac = 2^2 - 4 \cdot 1 \cdot 5 = 4 - 20 = -16 $$

Como $\Delta < 0$, não existem raízes reais. Isso ocorre porque não podemos extrair a raiz quadrada de um número negativo no conjunto dos reais ($\mathbb{R}$).

Portanto, a equação não possui solução real.



\subsection*{Soma e Produto}
O método da soma e produto, diferentemente da fórmula de Bhaskara, não retorna imediatamente as raízes da equação, mas fornece informações sobre as raízes, que devem ser estimadas com estas novas "dicas".

O método é recomendado quando as raízes são inteiras. Por mais paradoxal que possa soar, é comum que os vestibulares, ou exercícios de apostilas, tragam equações de segundo grau feitas propositalmente para que as raízes sejam inteiras, então o método é uma aposta válida, especialmente em tais contextos. Adicionalmente, com a prática, fica evidente quando as raízes serão facilmente encontradas ou não.

Sugere-se tentar o método da soma e produto inicialmente. Se a soma ou o produto não forem inteiros ou se não for possível encontrar as raízes em poucos segundos, recorra a um método mais certo, ainda que mais lento, como é o caso da fórmula de Bhaskara.

$$\begin{aligned}
    S =& \; x_1 + x_2 = -\frac{b}{a}  \\
    P =& \; x_1 \times x_2 = \phantom{-} \frac{c}{a}
\end{aligned}$$
Onde:
\begin{itemize}
    \item $a,b,c$ são os coeficientes da equação, assim como definidos inicialmente.
    \item $S$ é a soma das raízes
    \item $P$ é o produto das raízes
\end{itemize}

Há alguns casos particulares que podem facilitar a obtenção das raízes da equação:
\begin{itemize}
    \item Quando $P=0$, uma das raízes é $0$ e, consequentemente, a outra raiz é igual a $S$;
    \item Quando $S=0$, uma das raízes é $\sqrt{P}$ e a outra é $-\sqrt{P}$;
    \item Quando $S = P+1$, ou seja, $S$ é uma unidade maior do que $P$, uma das raízes é $1$ e a outra raiz é $P$;
    \item Quando $P$ é um quadrado perfeito e $S$ é par, verifique se $2\sqrt{P} =S$. Neste caso, as duas raízes são iguais a $\frac{S}{2}$, ou simplesmente $\sqrt{P}$.
\end{itemize}

Ao analisar a soma e o produto, você pode notar alguns padrões: o produto pode ser positivo ou negativo e a soma pode ser positiva ou negativa. Obviamente estamos excluindo os casos em que $S=0$ ou $P=0$, pois foram contemplados anteriormente.

\begin{center}
\begin{tabular}{|c|c|c|}
\hline
 & $P > 0$ & $P < 0$ \\
\hline
$S > 0$ & Ambas positivas & Uma positiva, uma negativa \\
\hline
$S < 0$ & Ambas negativas & Uma negativa, uma positiva \\
\hline
\end{tabular}
\end{center}


Dedique um tempo para compreender as relações estabelecidas acima. Mantenha em mente que, ao somar dois números, mantém-se o sinal do maior número (em módulo) e que a multiplicação de números com sinais iguais sempre é positiva e a multiplicação de números com sinais diferentes sempre é negativa.

Se a soma e o produto que você obteve não se encaixam em nenhum dos casos anteriores, recomenda-se fatorar $P$ em primos. E brincar de "montar" número de tal modo que a soma das possíveis raízes seja igual a $S$. É evidente que esse processo pode ser demorado, o que possivelmente indicará que uma mudança de estratégia é o melhor caminho a ser tomado.

A demonstração do método da soma e produto pode ser encontrada após a seção de exercícios.



\subsubsection*{Exemplo 2.1}

Considere novamente a equação:
$$ x^2 - 5x + 6 = 0 $$

Identificamos os coeficientes:
\begin{itemize}
    \item $a = 1$
    \item $b = -5$
    \item $c = 6$
\end{itemize}

Calculamos a soma e o produto das raízes:
\[
\begin{aligned}
    S &= -\frac{b}{a} = -\frac{-5}{1} = 5 \\
    P &= \frac{c}{a} = \frac{6}{1} = 6
\end{aligned}
\]

Agora, procuramos dois números cuja soma seja $5$ e cujo produto seja $6$. Os números que satisfazem isso são $2$ e $3$, pois:
\[
2 + 3 = 5 \quad \text{e} \quad 2 \cdot 3 = 6
\]

Portanto, as raízes da equação são $x_1 = 2$ e $x_2 = 3$.



\subsubsection*{Exemplo 2.2}

Considere a equação:
$$ x^2 + 5x = 0 $$

Primeiramente, completamos a equação geral:
$$ x^2 + 5x + 0 = 0 $$

Coeficientes:
\begin{itemize}
    \item $a = 1$
    \item $b = 5$
    \item $c = 0$
\end{itemize}

Calculamos a soma e o produto das raízes:
\[
\begin{aligned}
    S &= -\frac{b}{a} = -\frac{5}{1} = -5 \\
    P &= \frac{c}{a} = \frac{0}{1} = 0
\end{aligned}
\]

Quando o produto das raízes é zero ($P = 0$), uma das raízes obrigatoriamente é zero. A outra raiz será igual à soma, já que:
$$ x_1 + x_2 = -5 \quad \text{e} \quad x_1 \cdot x_2 = 0 $$

Logo, as raízes são:
\[
x_1 = 0, \quad x_2 = -5
\]

Portanto, a equação \( x^2 + 5x = 0 \) tem duas raízes reais: $0$ e $-5$.



\subsubsection*{Exemplo 2.3}

Considere a equação:
$$ x^2 - 36 = 0 $$

Podemos reescrevê-la como:
$$ x^2 + 0x - 36 = 0 $$

Coeficientes:
\begin{itemize}
    \item $a = 1$
    \item $b = 0$
    \item $c = -36$
\end{itemize}

Calculamos a soma e o produto das raízes:
\[
\begin{aligned}
    S &= -\frac{b}{a} = -\frac{0}{1} = 0 \\
    P &= \frac{c}{a} = \frac{-36}{1} = -36
\end{aligned}
\]

Quando $S = 0$, sabemos que as raízes são opostas (uma é o negativo da outra), e podemos usar:
\[
x_1 = \sqrt{36} = 6, \quad x_2 = -\sqrt{36} = -6
\]

Portanto, as raízes da equação são:
\[
x_1 = 6 \quad \text{e} \quad x_2 = -6
\]



\subsubsection*{Exemplo 2.4}

Considere a equação:
$$ x^2 + 16x - 17 = 0 $$

Coeficientes:
\begin{itemize}
    \item $a = 1$
    \item $b = 16$
    \item $c = -17$
\end{itemize}

Calculamos a soma e o produto das raízes:
\[
\begin{aligned}
    S &= -\frac{b}{a} = -\frac{16}{1} = -16 \\
    P &= \frac{c}{a} = \frac{-17}{1} = -17
\end{aligned}
\]

Note que:
\[
S = P + 1
\]

Esse é um caso particular conhecido: quando a soma das raízes é uma unidade maior do que o produto, uma das raízes será \( 1 \) e a outra será \( P \). Assim:

\[
x_1 = 1, \quad x_2 = -17
\]

Verificando:
$$
1 + (-17) = -16  $$
$$
1 \times (-17) = -17  
$$

Portanto, as raízes da equação são:
\[
x_1 = 1 \quad \text{e} \quad x_2 = -17
\]



\subsubsection*{Exemplo 2.5}

Considere a equação:
$$ x^2 - 10x + 25 = 0 $$

Coeficientes:
\begin{itemize}
    \item $a = 1$
    \item $b = -10$
    \item $c = 25$
\end{itemize}

Calculamos a soma e o produto das raízes:
\[
\begin{aligned}
    S &= -\frac{b}{a} = -\frac{-10}{1} = 10 \\
    P &= \frac{c}{a} = \frac{25}{1} = 25
\end{aligned}
\]

Note que:
\begin{itemize}
    \item $P = 25 = 5^2$ (um quadrado perfeito);
    \item $S = 10$ é par;
    \item $2 \cdot \sqrt{P} = 2 \cdot 5 = 10 = S$.
\end{itemize}

Portanto, as duas raízes são reais e iguais a:
\[
x_1 = x_2 = \frac{S}{2} = \frac{10}{2} = 5
\]




\pagebreak

\begin{center}
{\Large Exercícios}
\end{center}

A lista é extensa, nem todos os itens devem ser feitos na íntegra. Faça os itens até que você tenha certeza de que é capaz de fazer os demais itens sem dificuldades. 

Para ajudá-los, adotei um sistema para classificar a dificuldade dos exercícios que vai de \1 \ (fácil), passa por \2 , \3 , \4 \ e chega a \5 \ (muito difícil). Tal métrica não é absoluta e pode conter erros ou variar de pessoa para pessoa!

\vspace{5mm}

\begin{enumerate}
\item Resolva as equações abaixo usando a fórmula de Bhaskara. Ao final, verifique se as respostas, quando existirem, realmente são raízes da equação. \1 a \3

\begin{multicols}{2}
\begin{enumerate}[label=\arabic*.]
    \item $x^2 + 2x - 3 = 0$
    \item $x^2 + 10x + 25 = 0$
    \item $x^2 - 14x + 49 = 0$
    \item $x^2 + 7x - 30 = 0$
    \item $x^2 - 11x + 24 = 0$
    \item $x^2 - 6x + 5 = 0$
    \item $x^2 - 5x + 6 = 0$
    \item $5x^2 + 3x - 2 = 0$
    \item $4x^2 - 4x - 15 = 0$
    \item $x^2 + 4x - 21 = 0$
    \item $x^2 + 3x - 10 = 0$
    \item $2x^2 + 5x - 3 = 0$
    \item $x^2 - x - 12 = 0$
    \item $x^2 + 4x + 4 = 0$
    \item $x^2 - 8x + 16 = 0$
    \item $x^2 - 10x + 9 = 0$
    \item $x^2 - 2x - 8 = 0$
    \item $x^2 - 10x + 25 = 0$
    \item $x^2 - 36 = 0$
    \item $x^2 + 5x = 0$
    \item $x^2 + 16x - 17 = 0$
    \item $x^2 - 2x + 1 = 0$
    \item $x^2 - 6x + 9 = 0$
    \item $x^2 + 9x + 14 = 0$
    \item $3x^2 + 2x - 1 = 0$
    \item $2x^2 - 3x - 2 = 0$
    \item $x^2 + x + 1 = 0$
    \item $x^2 + 6x + 13 = 0$
    \item $3x^2 - 13x + 10 = 0$
    \item $x^2 - 4x - 12 = 0$
    \item $6x^2 - 5x - 6 = 0$
    \item $4x^2 - 4x + 1 = 0$
    \item $x^2 + 12x + 40 = 0$
    \item $x^2 + x + 4 = 0$
    \item $2x^2 - 7x + 3 = 0$
    \item $3x^2 - 12x + 12 = 0$
    \item $5x^2 - 20x + 20 = 0$
    \item $2x^2 + 8x + 8 = 0$
    \item $x^2 - x + 7 = 0$
    \item $x^2 + 2x + 5 = 0$
    \end{enumerate}
\end{multicols}

\newpage
\item Resolva as equações abaixo usando o método da Soma e Produto. Ao final, verifique se as respostas, realmente são raízes da equação.  \1 a \3
\begin{multicols}{2}
\begin{enumerate}[label=\arabic*.]
    \item $x^2 - 5x + 6 = 0$
    \item $x^2 + 7x + 10 = 0$
    \item $x^2 - 6x + 8 = 0$
    \item $x^2 - 9x + 20 = 0$
    \item $x^2 + 3x - 10 = 0$
    \item $x^2 - 3x - 10 = 0$
    \item $x^2 + 11x + 30 = 0$
    \item $x^2 + x - 12 = 0$
    \item $x^2 - x - 20 = 0$
    \item $x^2 + 4x - 21 = 0$
    \item $x^2 - 2x - 15 = 0$
    \item $x^2 + 6x + 5 = 0$
    \item $x^2 - 8x + 7 = 0$
    \item $x^2 - 10x + 21 = 0$
    \item $x^2 + 10x + 16 = 0$
    \item $x^2 + 13x + 40 = 0$
    \item $x^2 - 4x + 3 = 0$
    \item $x^2 - 11x + 30 = 0$
    \item $x^2 + 12x + 35 = 0$
    \item $x^2 - 13x + 36 = 0$
    \item $x^2 + 15x + 50 = 0$
    \item $x^2 + 14x + 48 = 0$
    \item $x^2 - 12x + 27 = 0$
    \item $x^2 - 15x + 50 = 0$
    \item $x^2 - 7x + 10 = 0$
    \item $x^2 + 8x + 7 = 0$
    \item $x^2 - 16x + 63 = 0$
    \item $x^2 + 9x + 14 = 0$
    \item $x^2 - 14x + 45 = 0$
    \item $x^2 + 5x - 24 = 0$
\end{enumerate}
\end{multicols}

\end{enumerate}

\pagebreak

\begin{center}
{\Large Demonstrações}
\end{center}
\subsection*{Fórmula de Bhaskara}

Queremos resolver a equação geral do segundo grau:
\[
ax^2 + bx + c = 0 \qquad (a \neq 0)
\]

\textbf{Passo 1:} Dividimos todos os termos por $a$ para deixar o coeficiente de $x^2$ igual a 1:
\[
x^2 + \frac{b}{a}x + \frac{c}{a} = 0
\]

\textbf{Passo 2:} Subtraímos $\frac{c}{a}$ dos dois lados:
\[
x^2 + \frac{b}{a}x = -\frac{c}{a}
\]

\textbf{Passo 3:} Completamos o quadrado. Somamos e subtraímos $\left(\frac{b}{2a}\right)^2$ do lado esquerdo:
\[
x^2 + \frac{b}{a}x + \left( \frac{b}{2a} \right)^2 = -\frac{c}{a} + \left( \frac{b}{2a} \right)^2
\]

\textbf{Passo 4:} Escrevemos o lado esquerdo como quadrado perfeito:
\[
\left( x + \frac{b}{2a} \right)^2 = \frac{b^2 - 4ac}{4a^2}
\]

\textbf{Passo 5:} Extraímos a raiz quadrada dos dois lados:
\[
x + \frac{b}{2a} = \pm \frac{\sqrt{b^2 - 4ac}}{2a}
\]

\textbf{Passo 6:} Isolamos $x$:
\[
x = \frac{-b \pm \sqrt{b^2 - 4ac}}{2a}
\]

Assim, obtemos a \textbf{fórmula de Bhaskara}:
\[
x = \frac{-b \pm \sqrt{\Delta}}{2a} \quad \text{com } \Delta = b^2 - 4ac
\]

\pagebreak
\subsection*{Soma e Produto}
Da fórmula de Bhaskara, temos:

$$
\begin{aligned}
x_1 &= \frac{-b + \sqrt{b^2 - 4ac}}{2a} \\
x_2 &= \frac{-b - \sqrt{b^2 - 4ac}}{2a}
\end{aligned}
$$

Para a soma:

\textbf{Passo 1:} Substituímos pelos valores que já conhecemos:
$$x_1+x_2 = \frac{-b + \sqrt{b^2 - 4ac}}{2a} + \frac{-b - \sqrt{b^2 - 4ac}}{2a}$$

\textbf{Passo 2:} Realizamos a soma de frações:
$$x_1+x_2 = \frac{(-b + \sqrt{b^2 - 4ac})+(-b - \sqrt{b^2 - 4ac})}{2a}$$

\textbf{Passo 3:} Removemos os parênteses:
$$x_1+x_2 = \frac{-b + \sqrt{b^2 - 4ac} -b - \sqrt{b^2 - 4ac})}{2a}$$

\textbf{Passo 4:} Juntamos os termos semelhantes:
$$x_1+x_2 = \frac{-2b}{2a}$$

\textbf{Passo 5:} Simplificamos a fração:
$$x_1+x_2 =\frac{-b}{a}$$

Assim, obtemos a \textbf{soma}:
\[
x_1+x_2 = \frac{-b}{a}
\]

\break

Para o produto:

\textbf{Passo 1:} Substituímos pelos valores que já conhecemos:
$$x_1 \times x_2 = \frac{-b + \sqrt{b^2 - 4ac}}{2a} \times \frac{-b - \sqrt{b^2 - 4ac}}{2a}$$

\textbf{Passo 2:} Realizamos a multiplicação de frações:
$$x_1 \times x_2 = \frac{(-b + \sqrt{b^2 - 4ac})\times(-b - \sqrt{b^2 - 4ac})}{(2a) \times (2a)}$$

\textbf{Passo 3:} No denominador, usamos o produto da soma pela diferença $(x+y)\times(x-y)=x^2-y^2$:
$$x_1 \times x_2 = \frac{(-b)^2 - \left(\sqrt{b^2 - 4ac}\right)^2}{(2a)^2}$$

\textbf{Passo 4:} Efetuamos as potenciações:
$$x_1 \times x_2 = \frac{b^2 - \left(b^2 - 4ac\right)}{4a^2}$$

\textbf{Passo 5:} Removemos os parênteses:
$$x_1 \times x_2 = \frac{b^2 - b^2 + 4ac}{4a^2}$$

\textbf{Passo 6:} Juntamos os termos semelhantes:
$$x_1 \times x_2 = \frac{4ac}{4a^2}$$

\textbf{Passo 7:} Simplificamos a fração:
$$x_1 \times x_2 = \frac{c}{a}$$

Assim, obtemos o \textbf{produto}:
\[
x_1 \times x_2 = \frac{c}{a}
\]

\end{document}

