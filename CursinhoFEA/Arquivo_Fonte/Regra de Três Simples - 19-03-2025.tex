\documentclass[12pt]{report}

\usepackage{graphicx} % Required for inserting images
\usepackage[brazil]{babel}
%\usepackage[latin1]{inputenc}
\usepackage{amsmath}
\usepackage{amsfonts}
\usepackage{amssymb}
\usepackage{parskip}
\usepackage{float}
\usepackage{mathtools}
\usepackage[autostyle]{csquotes}
\MakeOuterQuote{"}
\usepackage{geometry}
 \geometry{ a4paper, left=20mm, top=20mm, right=20mm, bottom=20mm}
\usepackage{systeme}
\usepackage{mathtools}
\usepackage{lipsum}
\usepackage{enumitem}
\usepackage{multicol}
\usepackage{multirow}
\usepackage{array}
\newcommand\mc[1]{\multicolumn{1}{c}{#1}}
\usepackage{xfrac}
\usepackage{nicematrix,tikz,booktabs}
\usepackage{fontawesome5} % termômetros
\renewcommand{\baselinestretch}{1.1}

\usepackage{fancyhdr}
\pagestyle{fancy}
\fancyhf{}
\fancyfoot[LO]{\footnotesize {Eduardo Yukio G. Ishihara\\ eduardoyukio.ishihara@usp.br}}
\fancyfoot[C]{\thepage}
\renewcommand{\headrulewidth}{0pt}
\renewcommand{\footrulewidth}{0pt}


\newcommand{\1}{\faThermometerEmpty}
\newcommand{\2}{\faThermometerQuarter}
\newcommand{\3}{\faThermometerHalf}
\newcommand{\4}{\faThermometerThreeQuarters}
\newcommand{\5}{\faThermometerFull}


\title{Lista}

\begin{document}

\begin{center}
{\Large Regra de Três Simples} \\ % ALTERAR
\vspace{1mm}
Eduardo Yukio G. Ishihara \\ 19/03/2025 \\ % ALTERAR
{\footnotesize eduardoyukio.ishihara@usp.br \\
Editado em 21/04/2025} % ALTERAR
\end{center}
\vspace{5mm}


\section*{Diretamente Proporcionais}
Dizemos que duas grandes são diretamente proporcionais quando o aumento de uma também acarreta no aumento de outra. De forma geral, multiplicar uma das grandezas faz com que a outra também seja multiplicada pelo mesmo fator.
Exemplos:
\begin{itemize}
    \item Quantidade de maçãs e o preço a ser pago por elas;
    \item Velocidade de um carro e a distância percorrida, fixado um tempo;
    \item Quantidade de estudantes e a quantidade de exercícios a serem corrigidos;
    \item Quantidade de átomos e a massa total;
    \item Quantidade de animais e a biomassa da espécie;
    \item Compressão de uma mola (em cm) e a força elástica;
    \item Massa de um soluto e concentração desse mesmo soluto.
\end{itemize}
Note que, definidas duas grandezas, não precisamos entender qual provoca o aumento de qual, isso é, a causa e a consequência, basta entender que as duas crescem (ou decrescem), sob um mesmo fator multiplicativo ($\times 3$, $\times 2$, $\div 2$, $\div 6$, etc).

Um critério essencial para afirmarmos que duas grandezas são diretamente proporcionais é garantirmos que, quando uma zera, a outra também zera! Tal critério é uma consequência da nota acima, pois um fator multiplicativo $\times 0$ ainda é um fator multiplicativo. Nos exemplos trazidos, sempre que uma das grandezas é zerada, a outra também será zerada — pode testar!

{\footnotesize A explicação a seguir utiliza conhecimentos de funções lineares, pode pular se não estiver familiarizado!}

Esse critério é essencial, pois entendemos as grandezas proporcionais como funções lineares (com o coeficiente linear igual a $0$ ($f(x)=ax+b, \ b:=0)$. Ao aplicarmos o algoritmo da regra de $3$, usamos as propriedades das funções lineares. Caso esse critério não seja seguido, ainda podemos fazer a análise, mas usando outros métodos, pois a regra de $3$ muito provavelmente levará a um resultado errôneo.


\section*{Inversamente Proporcionais}
Dizemos que duas grandezas são inversamente proporcionais quando o aumento de uma acarreta na redução da outra. De forma geral, multiplicar uma das grandezas faz com que a outra seja dividida pelo mesmo fator. Exemplos:
\begin{itemize}
    \item Quantidade de funcionários e o número de dias que leva para uma parede ser construída;
    \item Velocidade de um carro e o tempo que leva para percorrer uma mesma distância;
    \item Preço de uma maçã e a quantidade máxima que posso levar com um orçamento fixado;
    \item Massa de um solvente e concentração de um soluto;
    \item A  resistência e a corrente elétrica, fixado um potencial elétrico;
    \item Quantidade de bolas em uma urna e a probabilidade de uma bola fixada ser sorteada ao acaso.
\end{itemize}

Note que, mais uma vez, não é necessário estabelecer qual grandeza provoca o acréscimo ou decréscimo da outra, basta enxergar que as duas estão conectadas e crescem e descressem sob um mesmo fator multiplicativo. Por exemplo, multiplicar a quantidade de funcionários por $2$ faz com que o número de dias seja dividido por $2$. Ou seja, aplicamos a operação inversa da multiplicação, a divisão. Analogamente, dividir o número de funcionários por $3$ faz com o que a duração seja multiplicada por $3$.

\section*{Regra de $3$}
A tão conhecida "regra de $3$" nada mais é do que um mecanismo criado para facilitar o cálculo de proporções diretas. De forma geral, são fornecidas duas grandezas, uma relação entre elas (direta ou inversamente proporcionais) e três valores e o objetivo final é determinar o quarte valor faltante. Por questões didáticas, é usual transformar tal processo numa tabela.

\textbf{Exemplo 1}: Uma pessoa vai ao mercado e deseja comprar barras de chocolate. Uma nova promoção diz que cada barra de chocolate sai por R\$ $5,50$. Uma pessoa deseja comprar $18$ barras de chocolate. Quanto ela pagará ao final da compra?

Podemos resumir os dados na seguinte tabela:

\begin{center}
\begin{tabular}{|c | c|} 
 \hline
 Barras de chocolate & Preço (R\$)  \\ [0.5ex] 
 \hline \hline
 $1$ & $5,50$ \\ \hline
 $18$ & $x$ \\ 
 \hline
\end{tabular}
\end{center}

Agora, precisamos determinar a relação entre as grandezas. Se dobrarmos o número de barras de chocolate, o preço e dobrado ou dividido por dois? Evidentemente o preço é dobrado. Portanto, são grandezas diretamente proporcionais (quando uma cresce, a outra também cresce). Setas apontando para a mesma direção são diretamente proporcionais e setas apontando para direções opostas são inversamente proporcionais.

\begin{center}
\begin{tabular}{|c | c|} 
 \hline
 Barras de chocolate & Preço (R\$)  \\ [0.5ex] 
 \hline \hline
 $1$ & $5,50$ \\ \hline
 $\uparrow$ &  $\uparrow$ \\ \hline
 $18$ & $x$ \\ 
 \hline
\end{tabular}
\end{center}

Definidas que as duas grandezas são diretamente proporcionais, podemos prosseguir para o procedimento clássico: multiplicação em cruz. Temos a seguinte equação:
$$
1 \times x = 18 \times 5,50 \Rightarrow x = 99,00
$$
Portanto, conclui-se que a pessoa pagará R\$$99,00$ pelas $18$ barras de chocolate.

\vspace{10pt}
\textbf{Exemplo 2}: Um carro percorre, em $2$ horas uma distância de $150Km$. Mantendo essa mesma velocidade, qual distância o carro percorrerá em um período de $9$ horas?

Repetiremos o mesmo processo do exemplo anterior:

\begin{center}
\begin{tabular}{|c | c|} 
 \hline
 Tempo (horas) & Distância ($Km$)  \\ [0.5ex] 
 \hline \hline
 $2$ & $150$ \\ \hline
 $9$ & $x$ \\ 
 \hline
\end{tabular}
\end{center}

Com o aumento do tempo e velocidade constante, é natural que a distância percorrida também seja aumentada, ou seja, são diretamente proporcionais.

\begin{center}
\begin{tabular}{|c | c|} 
 \hline
 Tempo (horas) & Distância ($Km$)  \\ [0.5ex] 
 \hline \hline
 $2$ & $150$ \\ \hline
 $\uparrow$ &  $\uparrow$ \\ \hline
 $9$ & $x$ \\ 
 \hline
\end{tabular}
\end{center}

Por fim, multiplicamos em cruz:

$$
2 \times x = 9 \times 150 \Rightarrow 2x = 1350 \Rightarrow x = \frac{1350}{2} = 675
$$
Portanto, conclui-se que o carro percorrerá $675Km$ em $9$ horas.

\vspace{10pt}
\textbf{Exemplo 3}: Uma construtora tem $20$ funcionários e constrói um galpão em $8$ semanas. Após um corte de funcionários, apenas $14$ funcionários foram mantidos. Qual é o tempo que a construtora levará para construir um galpão equivalente ao primeiro?

Resumindo os dados na tabela:

\begin{center}
\begin{tabular}{|c | c|} 
 \hline
 Funcionários & Duração (semanas) \\ [0.5ex] 
 \hline \hline
 $20$ & $8$ \\ \hline
 $14$ & $x$ \\ 
 \hline
\end{tabular}
\end{center}

Note que, nesse novo exemplo, aumentar o número de funcionários faz com que a duração diminua. Ou seja, as grandezas são inversamente proporcionais, então as setas terão direções opostas.

\begin{center}
\begin{tabular}{|c | c|} 
 \hline
 Funcionários & Duração (semanas) \\ [0.5ex] 
 \hline \hline
 $20$ & $8$ \\ \hline
 $\uparrow$ &  $\downarrow$ \\ \hline
 $14$ & $x$ \\ 
 \hline
\end{tabular}
\end{center}

Para proceder com os casos inversamente proporcionais, vamos criar uma "tabela auxiliar", onde inverteremos os valores cujas setas apontam para baixo:

\begin{center}
{\footnotesize Tabela Auxiliar} \\
\begin{tabular}{|c | c|} 
 \hline
 Funcionários & Duração (semanas) \\ [0.5ex] 
 \hline \hline
 $20$ & $x$ \\ \hline
 $\uparrow$ &  $\uparrow$ \\ \hline
 $14$ & $8$ \\ 
 \hline
\end{tabular}
\end{center}

Agora que a tabela auxiliar está montada e todos os valores têm setas apontando para a mesma direção, podemos prosseguir para o procedimento padrão e multiplicar em cruz:
$$
20 \times 8 = 14 \times x \Rightarrow 160 = 14 x \Rightarrow x = \frac{160}{14} = \frac{80}{7} \approx 14.43
$$
Portanto, a construtora levará \sfrac{80}{7}, ou aproximadamente $14.43$ semanas para terminar a obra com o novo quadro de funcionários.

\vspace{10pt}
\textbf{Exemplo 4}: Uma fazendeira tem um silo capaz de armazenar alimento suficiente para alimentar os seus $32$ porcos por $5$ meses. Após o período de vendas, apenas $18$ porcos permanecera na sua propriedade. Nesta nova condição, um silo é capaz de armazenar alimento suficiente para quantos meses?

Reduzindo os dados na tabela:

\begin{center}
\begin{tabular}{|c | c|} 
 \hline
 Porcos & Tempo (meses) \\ [0.5ex] 
 \hline \hline
 $32$ & $5$ \\ \hline
 $18$ & $x$ \\ 
 \hline
\end{tabular}
\end{center}

É fácil perceber que quanto mais porcos, menos tempo dura o alimento armazenado do silo. Então, são grandezas inversamente proporcionais:

\begin{center}
\begin{tabular}{|c | c|} 
 \hline
 Porcos & Tempo (meses) \\ [0.5ex] 
 \hline \hline
 $32$ & $5$ \\ \hline
 $\uparrow$ &  $\downarrow$ \\ \hline
 $18$ & $x$ \\ 
 \hline
\end{tabular}
\end{center}

Uma vez que as setas não apontam para a mesma direção, constrói-se a tabela auxiliar invertendo uma das colunas:


\begin{center}
{\footnotesize Tabela Auxiliar} \\
\begin{tabular}{|c | c|} 
 \hline
 Porcos & Tempo (meses) \\ [0.5ex] 
 \hline \hline
 $32$ & $x$ \\ \hline
 $\uparrow$ &  $\uparrow$ \\ \hline
 $18$ & $5$ \\ 
 \hline
\end{tabular}
\end{center}

Finalmente, multiplica-se em cruz:
$$
32 \times 5 = 18 \times x \Rightarrow 160 = 18x \Rightarrow x = \frac{160}{18} = \frac{80}{9} \approx 8.89
$$
Portanto, o silo é capaz de armazenar alimento para $18$ porcos por \sfrac{80}{9}, ou aproximadamente, $8.89$ meses.



\pagebreak

\begin{center}
{\Large Exercícios}
\end{center}

A lista é extensa, nem todos os itens devem ser feitos na íntegra. Faça os itens até que você tenha certeza de que é capaz de fazer os demais itens sem dificuldades. 

Para ajudá-los, adotei um sistema para classificar a dificuldade dos exercícios que vai de \1 \ (fácil), passa por \2 , \3 , \4 \ e chega a \5 \ (muito difícil). Tal métrica não é absoluta e pode conter erros ou variar de pessoa para pessoa!

\vspace{5mm}

\begin{enumerate}
\item Determine se as grandezas abaixo são proporcionais, inversamente proporcionais ou nenhum dos dois. Não é necessária verificar se são diretamente, isso é, quando uma zera, a outra também zera. Quando não houver relação direta, explique por que ela não existe. \1 a \2

\begin{enumerate}
    \item Quantidade de alunos e o número de carteiras numa sala;
    \item Quantidade de alunos e a atenção que um professor pode dar para cada aluno;
    \item Quantidade de árvores e a área com sombra;
    \item Ano de nascimento e a quantidade de filhos;
    \item Temperatura de um pote e o tempo que ficou no fogão;
    \item Temperatura de um pote e o tempo que ficou no refrigerador;
    \item Velocidade de uma reação e a concentração de um catalisador;
    \item Velocidade de uma reação e a concentração dos reagentes;
    \item Força gravitacional e a massa dos corpos;
    \item Quantidade de copos e o volume de um dos copos;
    \item Força gravitacional e a distância entre os corpos;
    \item Altura e o IMC;
    \item Massa e o IMC;
    \item Massa de um doce e quantidade de açúcar no doce;
    \item Massa de um doce e concentração de açúcar no doce;
    \item Volume de um cubo de metal e a massa desse mesmo cubo;
    \item Volume de um cubo de metal e a densidade desse mesmo cubo;
    \item Altura de uma pessoa e a quantidade de sapatos;
    \item Concentração de $O_2$ e altitude em relação ao mar;
    \item Temperatura de ebulição da água e altitude em relação ao mar;
    \item Tempo de exposição a um material radioativo e chances de mutação em uma célula;
    \item Tempo desde a aplicação de um remédio e sua concentração no sangue;
    \item Volume de água ingerido e quantidade de fios de cabelo.
    \end{enumerate}


\item Nos itens abaixo, determine se as grandezas são diretamente ou inversamente proporcionais e resolva as situações problemas. \1 a \3
\begin{enumerate}
\item Para alimentar o seu cão, uma pessoa gasta $10\ kg$ de ração a cada $15$ dias. Qual a quantidade total de ração consumida por semana, considerando que, por dia, é sempre colocada a mesma quantidade de ração?  
\item Uma torneira enche um tanque em $5$ horas. Quanto tempo levaria para $3$ com a mesma vazão da torneira anterior  encherem esse mesmo tanque?
\item Em uma escola, cada professor corrige, em média, $15$ páginas de lição por aluno a cada mês. Uma turma, que continha $13$ alunos, passou a ter $16$ alunos com a chegada de $3$ alunos transferidos. Em média, quantas páginas o professor dessa turma passará a corrigir a cada mês?
\item Em uma empresa, uma máquina produz $25$ milhares de unidades de parafuso por semana. O gerente deseja expandir a produção para mais regiões do país e comprará mais $7$ máquina idênticas à primeira. Qual é a produção esperada por semana após a expansão?
\item Uma concorrente da empresa leva $15$ dias para produzir $150$ milhares de unidades de parafuso. Eles desejam reduzir esse tempo para, no máximo, $4$ dias. Quantas máquinas eles devem comprar para atender o novo prazo? Lembre-se que não é possível comprar meia máquina, arredonde apropriadamente a sua resposta.
\item O ângulo externo de um polígono regular é inversamente proporcional à quantidade de lados. Se um polígono de $3$ lados tem ângulos internos iguais a $120 $\textdegree, determine o ângulo externo de um polígono de $7$ lados.
\item \3 Usando o mesmo cenário do item anterior, determine o ângulo externo de um polígono de $n$ lados. A sua resposta deve ser a fórmula para o ângulo externo de um polígono regular, ou seja, isole o ângulo em função da quantidade de lados. Lembre-se que a soma dos ângulos externos de um polígono sempre é 360\textdegree.
\\ Dica: pense no que acontece se dobrarmos o número de lados.
\item Na fase final de um jogo, o inimigo causa dano ao jogador de maneira inversamente proporcional à distância entre o jogador e o inimigo. A uma distância de $6m$, o jogador leva $150$ pontos de dano. Qual deve ser a distância para o dano ser menor que $100$ pontos? E menor que $50$ pontos? E menor que $10$ pontos?
\item Ainda que pouco usado no cotidiano, define-se $1\ hm$ (lê-se "hectômetro") como $100\ m$. Converta $5\ hm$, $6.7\ hm$, $13.1\ hm$ e $50\ hm$ em metros. Agora, converta $1\ m$, $150\ m$, $6548\ m$, $1500\ m$ e $0.1\ m$ em $hm$.
\item $1m^3$ de água é equivalente a $1.000L$ de água. Determine quantos litros de água equivalem a $8m^3$ de água. Escreva com suas palavras uma maneira de converter $m^3$ em $L$.
\item Em uma cidade, o custo da conta de água é determinado pelo gasto de água em $m^3$ da residência. Um novo morador deseja encher sua piscina de $28.000\ L$, mas não sabe quanto custa o $m^3$ de água na cidade. Seu vizinho informou que, no mês passado, gastou $15\ m^3$ e pagou um valor de $R\$35,00$. Quanto custará ao novo morador encher sua piscina?
\item A bula de um medicamento determina que uma dose de $130\ mg$ deve ser administrada para cada $Kg$ de um paciente. Quantos $mg$ desse medicamento devem ser ministrados a um paciente de $70\ Kg$? E um de $120\ Kg$? Uma dose de $12.700\ mg$ é adequada para um paciente de, aproximadamente, quantos $Kg$?
\item Em estimativas oficiais, costuma-se adotar que uma multidão densa tem uma média de $4$ pessoas por $m^2$. Em um show recente num estádio em SP, o corpo de bombeiros verificou que estava muito lotado e, portanto, trata-se de uma multidão densa. O espaço ocupado pela multidão era a metade de um círculo de raio $40m$. Estime a quantidade de pessoas assistindo ao show.
\item Durante a realização de exames de sangue, coleta-se uma amostra do sangue da pessoa e, a partir dessa parcela, estima-se os parâmetros desejados de todo o sangue na pessoa. Tal técnica é utilizado, pois é inviável medir os parâmetros necessários de todo o sangue de uma pessoa. Um laboratório verificou que havia $11$ milhões de glóbulos brancos numa amostra de $1 \ ml$ de sangue de um homem adulto. Considerando que um homem adulto possui, em média, $5 \ L$ de sangue, estime a quantidade total de glóbulos brancos no sangue dessa pessoa.
\item Um carro leva $6$ horas para ir do Rio de Janeiro a São Paulo e consome $40\ L$ a cada viagem. Um funcionário de uma empresa precisa fazer essa viagem (ida e então a volta) duas vezes por mês. Considere que a volta (São Paulo para o Rio de Janeiro) leva o mesmo tempo e consumo o mesmo combustível. Determine o consumo de gasolina em uma ano.
\item Ainda nas condições do item anterior, determine quantas horas o motorista passou dirigindo de uma cidade para a outra no período de um ano.
\item Se a velocidade média do carro fosse dobrada, quantas horas o motorista teria passado dirigindo de uma cidade para a outra?
\item Um objeto percorre uma distância de $10\ m$ em $1\ s$. Mantendo essa mesma velocidade, qual seria a distância percorrida em $15.8\ s$?
\item \2 Ainda nas condições do item anterior, qual seria a distância percorrida se o tempo que o objeto ficou em movimento fosse $t$? Ou seja, monte a "regra de $3$", trabalhe com as variáveis e determine a distância em função do tempo.
\item \3 Um objeto percorre uma distância de $d$ em $1\ s$. Qual distância esse objeto percorreria em um tempo $t$? Ou seja, monte a "regra de $3$", trabalhe com as variáveis e determine a distância percorrida em função de $t$ e $d$. Não confunda as variáveis, chame a nova distância percorrida de uma nova letra. Resolver esse item é equivalente a deduzir a fórmula da velocidade média de um corpo. 

\end{enumerate}


\end{enumerate}
\end{document}
